\documentclass{cmc}

\begin{document}

\pagestyle{fancy}
\lhead{\textit{\textbf{Computational Motor Control, Spring 2019} \\
    Webots Tutorial, Lab 7, UNGRADED}} \rhead{Student \\ Names}

\section*{Webots mini tutorial}

\textit{Instructions: This document contains the basic information
  needed to set-up and get your self familiarized with Webots.  A
  physics simulator which we will be using during the remaining part
  of the course.}

\section{Webots Installation}
\label{sec:webots-installation}

Follow the
\href{https://www.cyberbotics.com/#download}{\corr{link}}
to find the instructions necessary to download and install
the latest version of Webots.

\section{Start Webots}
\label{sec:start-webots}

After starting the Webots, choose the Guided Tour (which should be
offered the first time you start Webots, otherwise you can also start
it from the menu Help $\rightarrow$ Webots Guided Tour). Please go
through all the Guided Tour examples and have a look at the short
descriptions in the window. You can Stop, Run and Revert the examples
by using the corresponding toolbar buttons. Please do try all these
buttons:
\begin{itemize}
\item \textbf{Revert} : Reload the .wbt file and restarts the
  simulation from the beginning
\item \textbf{Step} : Go one simulation step forward (physics)
\item \textbf{Real-time} : Run the simulation in real-time
\item \textbf{Run} : Run the simulation at maximal speed with the
  visualization
\item \textbf{Fast} : Run the simulation at the maximal speed allowed
  by the CPU power (OpenGL rendering is disabled for better
  performance)
\end{itemize}

Note that next to the toolbar buttons there are two indicators: The
left indicator shows the simulation elapsed time. The value is
displayed as Hours:Minutes:Seconds:Milliseconds. The elapsed time
indicates the time, it stops when the simulation is stopped. The other
indicator is the speedometer which indicates how fast the simulation
is currently running with respect to real time. When the speedometer
shows about 1x, this means that the speed of the displayed simulated
bodies is approximately equal to the speed that real physical bodies
would have.

\section{Navigate in Webots}
\label{sec:navigate-webots}

Using some of the examples of the Guided Tour, learn how to navigate
in 3D using the mouse: \textbf{To do}: Try pressing each of the mouse
buttons and the wheel while dragging the mouse left/right/up/down.

It is also possible to move objects using the mouse. This can be
achieved while the simulation is stopped or running. In order to move
an object: first select the object with a left mouse click, then hold
down the shift key and use the mouse:
\begin{itemize}
\item Translation: press the left mouse button to shift solid objects
  in the xz-plane (parallel to the ground).
\item Lift: press both left and right mouse buttons (or the middle
  mouse button), and move the mouse up/down to lift/lower a solid
  object.
\item Axis-aligned handles: When a solid object is selected, some
  arrow-shaped handles appear in the 3D window. These handles can be
  used to translate and rotate the object along the corresponding axis
\end{itemize}

To apply a force to an object, place the mouse pointer where the force
will apply, hold down the Alt key and left mouse button together while
dragging the mouse. Linux users should also hold down the Control key
(Ctrl) together with the Alt key.

\section{Scene Tree}
\label{sec:scene-tree}

In Webots, the Scene Tree is the window that describes the world model
in a VRML-like language. Geometrical shapes and dimensions, color and
textures, physical properties ... all these different parameters are
defined in the Scene Tree. As an example you can try to modify the
gravity in some of the examples of the Guided Tour.  For example, open
one of these worlds: Ghostdog, Gantry Robot or Q-RIO Robot
\begin{itemize}
\item Open the Scene Tree window if it is not opened yet (Menu Tools
  $\rightarrow$ SceneTree)
\item In the Scene Tree, expand the topmost node: WorldInfo
\item In the WorldInfo node: select the gravity field
\item In the gravity field: change the y gravity component from -9.81
  to 0.
\item Run the simulation: now your simulation should be running in
  zero-gravity!
\end{itemize}

The field WorldInfo $\rightarrow$ basicSimulationStep specifies the
duration of the simulation step in milliseconds. The simulation step
defines how much to advance in simulated time before recomputing
(integrating) the physics forces. Rigid bodies simulation is like
integration: The smaller the simulation step, the more accurate the
simulation, the larger the simulation step, the faster the
simulation. If the simulation step is too large, unrealistic behavior
may arise, for example: instability (vibration or explosion) of the
simulation, robot passing though the ground or walls, etc. If the step
is too small the simulation becomes too slow and you may have to wait
too long for your results. Because there is this trade off between CPU
time and faithfulness, it is important to always adjust the simulation
step.

\textbf{To do}: Open some of the worlds of Guided Tour and try to
increase / decrease the simulation step and Run the simulation
again. What do you notice?

\section{Basic concepts in Webots}
\label{sec:basic-conc-webots}

Please read sections Getting Started with Webots and Language Setup
Webots User Guide (Help $\rightarrow$ User Guide) to understand the
basic concepts of Webots (e.g. controllers, worlds, physics).

\textbf{TO DO : } Go to File $\rightarrow$ Open Sample World
$\rightarrow$ and search for python in search field and open
example.wbt file. If the search fails, in sample world window navigate
to languages $\rightarrow$ python $\rightarrow$ example.wbt.
Explore the example file and use the concepts learnt earlier to get
yourself familiar with Webots and Python.


\end{document}

%%% Local Variables:
%%% mode: latex
%%% TeX-master: t
%%% End:
